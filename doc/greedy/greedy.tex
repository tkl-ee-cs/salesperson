\documentclass{article}

\usepackage{algorithm}
\usepackage{algpseudocode}

\usepackage{enumitem}
\usepackage{amsmath, amssymb}

\usepackage[section]{placeins}

\setlength\parindent{0pt}

\begin{document}
\section{Greedy Heuristic}

Another straightforward approach is the Greedy approach heuristic. The greedy
heuristic works by evaluating all the edges to slowly build the tour. First the
edges are all sorted and the shortest edges is identified. The approach begins
by adding this smallest edge to a possible tour. It then finds the next shortest
edges and evaluates it. Does adding this new edge to the tour create a cycle
with less than $V$ edges? Does the newly added edge increase the degree of the
node to more than 2? Has this edge been used before? The approach checks each
edge with these criteria and as long as it does not violate them will add it the
tour. It will repeat this process until all of the nodes have a degree of two
and there are $|V|$ edges. This means the tour has created a Hamiltonian cycle.


\subsection{Pseudocode}

\FloatBarrier

\begin{algorithm}
  \caption{Greedy}
  \label{alg_greedy}
  \begin{algorithmic}[1]
    \Procedure {Greedy}{$\{ G=(V,E), S \}$}

    \State $E_{\textit{sorted}} \gets \textit{Sort}(E)$

    \State $x \gets \min E$

    \State $T \gets T \cup {x}$
    \\
    \While {$|V| \neq |T|$}
      \State $x \gets \min E_{\textit{sorted}}$
      \If {$x \cup T \textit{ has no cycle with edges } > |V|$}
        \If {$\textit{deg}(x) \leq 2$}
          \If {$\nexists x \in T$}
            \State $T \gets T \cup x$
          \EndIf
        \EndIf
      \EndIf
    \EndWhile
    \\
    \EndProcedure
  \end{algorithmic}
\end{algorithm}

\FloatBarrier

\subsection{Detail}
A major drawback to this approach is that it does no forecasting and picks the best selection at
the moment. This greedy shortsightedness is exactly like nearest neighbor, as they both do not
look ahead. The difference between the two greedy approaches is where the focus lies. The
nearest neighbor evaluates adjacent nodes, while the greedy approaches focuses on all edges to
build a tour piece by piece. As one can see it has a runtime of $O(n^2 log2(n))$[1] where a large
portion of the running time is spent ordering the edges. Like any greedy algorithm, it often
produces sub-optimal tours and requires additional improvement heuristics and algorithms to
increase quality. For all its simplicity, it also was not a good choice due to its inability to
approximate an optimal tou

\end{document}
